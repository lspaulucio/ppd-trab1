\documentclass[
	% -- opções da classe memoir --
	12pt,				% tamanho da fonte
	% openright,			% capítulos começam em pág ímpar (insere página vazia caso preciso)
    oneside,			% para impressão somente frente. Oposto a twoside (frente e verso)
	a4paper,			% tamanho do papel. 
	% -- opções da classe abntex2 --
	%chapter=TITLE,		% títulos de capítulos convertidos em letras maiúsculas
	%section=TITLE,		% títulos de seções convertidos em letras maiúsculas
	%subsection=TITLE,	% títulos de subseções convertidos em letras maiúsculas
	%subsubsection=TITLE,% títulos de subsubseções convertidos em letras maiúsculas
	% -- opções do pacote babel --
	english,			% idioma adicional para hifenização
	%french,				% idioma adicional para hifenização
	%spanish,			% idioma adicional para hifenização
	brazil,				% o último idioma é o principal do documento
	]{abntex2}


% ---
% PACOTES
% ---

% ---
% Pacotes fundamentais 
% ---
\usepackage{cmap}				% Mapear caracteres especiais no PDF
\usepackage{lmodern}			% Usa a fonte Latin Modern
%\usepackage{helvet}			% Usa a fonte helvet(ARIAL)
%\usepackage{pslatex}			
\usepackage[T1]{fontenc}		% Selecao de codigos de fonte.
\usepackage[utf8]{inputenc}		% Codificacao do documento (conversão automática dos acentos)
\usepackage{indentfirst}		% Indenta o primeiro parágrafo de cada seção.
\usepackage{color}				% Controle das cores
\usepackage{graphicx}			% Inclusão de gráficos
\usepackage{enumerate}

% ---
% Pacotes adicionais, usados no anexo do modelo de folha de identificação
% ---
\usepackage{multicol}
\usepackage{multirow}
% ---

% Permite incluir listagens de código com o comando \lstinputlisting{}.
\usepackage{listings}
\usepackage{caption}
\DeclareCaptionFont{white}{\color{white}}
\DeclareCaptionFormat{listing}{\colorbox{gray}{\parbox{\textwidth}{#1#2#3}}}
\captionsetup[lstlisting]{format=listing,labelfont=white,textfont=white}
\renewcommand{\lstlistingname}{Listagem}
\definecolor{mygray}{rgb}{0.5,0.5,0.5}
\lstset{
	basicstyle=\scriptsize,
	breaklines=true,
%	numbers=left,
	numbersep=5pt,
	numberstyle=\tiny\color{mygray}, 
	rulecolor=\color{black},
	showstringspaces=false,
	tabsize=4,
    inputencoding=utf8,
    extendedchars=true,
    literate=%
    {é}{{\'{e}}}1
    {è}{{\`{e}}}1
    {ê}{{\^{e}}}1
    {ë}{{\¨{e}}}1
    {É}{{\'{E}}}1
    {Ê}{{\^{E}}}1
    {û}{{\^{u}}}1
    {ù}{{\`{u}}}1
    {â}{{\^{a}}}1
    {à}{{\`{a}}}1
    {á}{{\'{a}}}1
    {ã}{{\~{a}}}1
    {Á}{{\'{A}}}1
    {Â}{{\^{A}}}1
    {Ã}{{\~{A}}}1
    {ç}{{\c{c}}}1
    {Ç}{{\c{C}}}1
    {õ}{{\~{o}}}1
    {ó}{{\'{o}}}1
    {ô}{{\^{o}}}1
    {Õ}{{\~{O}}}1
    {Ó}{{\'{O}}}1
    {Ô}{{\^{O}}}1
    {î}{{\^{i}}}1
    {Î}{{\^{I}}}1
    {í}{{\'{i}}}1
    {Í}{{\~{Í}}}1
}

	
% ---
% Pacotes adicionais, usados apenas no âmbito do Modelo Canônico do abnteX2
% ---
\usepackage{lipsum}				% para geração de dummy text
% ---

% ---
% Pacotes de citações
% ---
\usepackage[brazilian,hyperpageref]{backref}	 % Paginas com as citações na bibl
\usepackage[alf]{abntex2cite}	% Citações padrão ABNT

% --- 
% CONFIGURAÇÕES DE PACOTES
% --- 

% ---
% Configurações do pacote backref
% Usado sem a opção hyperpageref de backref
\renewcommand{\backrefpagesname}{Citado na(s) página(s):~}
% Texto padrão antes do número das páginas
\renewcommand{\backref}{}
% Define os textos da citação
\renewcommand*{\backrefalt}[4]{
	\ifcase #1 %
		Nenhuma citação no texto.%
	\or
		Citado na página #2.%
	\else
		Citado #1 vezes nas páginas #2.%
	\fi}%
% ---

% ---
% Informações de dados para CAPA e FOLHA DE ROSTO
% ---
\titulo{Relatório do 1º Trabalho de Processamento Paralelo e Distribuído}
\autor{Leonardo Santos Paulucio}
\local{Vitória - ES}
\data{22 de Maio de 2018}
\instituicao{%
  Universidade Federal do Espírito Santo
  %\par
  %Setor Palotina
  \par
  Engenharia da Computação}
\tipotrabalho{Relatório técnico}
% O preambulo deve conter o tipo do trabalho, o objetivo, 
% o nome da instituição e a área de concentração 
\preambulo{Trabalho apresentado à disciplina de Processamento Paralelo e Distribuído do curso Engenharia da Computação da Universidade Federal do Espírito Santo como requisito parcial de avaliação.
\newline \newline \textbf{Professor:} João Paulo A. Almeida}

% ---

% ---
% Configurações de aparência do PDF final

% alterando o aspecto da cor azul
%\definecolor{blue}{RGB}{41,5,195}
\definecolor{blue}{RGB}{0,0,0}

% informações do PDF
\makeatletter
\hypersetup{
     	%pagebackref=true,
		pdftitle={\@title}, 
		pdfauthor={\@author},
    	pdfsubject={\imprimirpreambulo},
	    pdfcreator={LaTeX with abnTeX2},
		pdfkeywords={abnt}{latex}{abntex}{abntex2}{relatório técnico}, 
		colorlinks=true,       		% false: boxed links; true: colored links
    	linkcolor=blue,          	% color of internal links
    	citecolor=blue,        		% color of links to bibliography
    	filecolor=magenta,      		% color of file links
		urlcolor=blue,
		bookmarksdepth=4
}
\makeatother
% --- 

% --- 
% Espaçamentos entre linhas e parágrafos 
% --- 

% O tamanho do parágrafo é dado por:
\setlength{\parindent}{1.3cm}

% Controle do espaçamento entre um parágrafo e outro:
\setlength{\parskip}{0.2cm}  % tente também \onelineskip

% ---
% compila o indice
% ---
\makeindex
% ---

% ----
% Início do documento
% ----
\begin{document}

% Retira espaço extra obsoleto entre as frases.
\frenchspacing 

% ----------------------------------------------------------
% ELEMENTOS PRÉ-TEXTUAIS
% ----------------------------------------------------------
% \pretextual

% ---
% Capa
% ---
\imprimircapa
% ---

% ---
% Folha de rosto
% (o * indica que haverá a ficha bibliográfica)
% ---
\imprimirfolhaderosto*
% ---

% ---
% RESUMO
% ---

% resumo na língua vernácula (obrigatório)
% \begin{resumo} %% AQUI COMEÇA A PÁGINA DE RESUMO
% Costuma-se dizer que, numa economia capitalista, os problemas econômicos relativos à decisão sobre que tipos de produtos devem ser produzidos e a que preços serão vendidos esses produtos são resolvidos normalmente pelo livre jogo das forças de mercado – isto é, pelo livre funcionamento da oferta e da demanda. Nesta hipótese, as decisões e escolhas econômicas são individualizadas e feitas pelos consumidores – que são os demandantes dos bens e serviços – e pelos produtores – que são os ofertantes. Agindo de acordo com seus próprios interesses, os indivíduos, afetando e sendo afetados pelo sistema de preços, tomam as decisões que maximizarão a satisfação coletiva. 
% O objetivo é o de explicar de maneira simplificada como atua um sistema de preços e sua influência na alocação de recursos escassos.
% Ocorre, porém, que a determinação do preço e da quantidade produzida de um bem ou serviço depende essencialmente do número de agentes econômicos – demandantes e ofertantes – existentes nesse mercado. Por isso, é interessante caracterizar, antes, os diversos tipos de mercado existentes.
% O mercado, como você sabe, é o local onde se encontram os vendedores e compradores de determinados bens e serviços. Antigamente, a palavra mercado tinha uma conotação estritamente geográfica, mas isso já está deixando de ser assim. Hoje, com os avanços tecnológicos nas comunicações, as transações econômicas podem se realizar sem contato pessoal direto entre comprador e vendedor, tal como ocorre nas compras e vendas pela internet.


%  \vspace{\onelineskip}
    
%  \noindent
%  \textbf{Palavras-chaves}: latex. abntex. editoração de texto.
% \end{resumo} %AQUI TERMINA A PÁGINA DE RESUMO
% ---

% ---
% inserir lista de ilustrações
% ---

%\listoffigures* %% o * indica que não será incluso no sumário
%\cleardoublepage %% Pula página
% ---

% ---
% inserir lista de tabelas
% ---

%\listoftables*
%\cleardoublepage
% ---

% ---
% inserir lista de abreviaturas e siglas
% ---
%\begin{siglas}
%  \item[Fig.] Area of the $i^{th}$ component
%  \item[456] Isto é um número
%  \item[123] Isto é outro número
%  \item[lauro cesar] este é o meu nome
%\end{siglas}
% ---

% ---
% inserir lista de símbolos
% ---
%\begin{simbolos}
%  \item[$ \Gamma $] Letra grega Gama
%  \item[$ \Lambda $] Lambda
%  \item[$ \zeta $] Letra grega minúscula zeta
%  \item[$ \in $] Pertence
%\end{simbolos}
% ---

% ---
% inserir o sumario
% ---

\tableofcontents*

% ---

% ----------------------------------------------------------
% ELEMENTOS TEXTUAIS  (necessário para incluir número nas páginas)
% ----------------------------------------------------------
\textual


% MONTELLA. MAURA. Micro e Macroeconomia: Uma Abordagem Conceitual e Prática. Edição nº 01 - Editora Atlas, São Paulo – SP, 2009.

% MANKIW, N. Gregory. Princípios da Microeconomia. Tradução Edição n.º 03 Norte-Americana – Editora Pioneira Thomson Learning, 2005.

% BEGG. David K. H. Introdução à Microeconomia. Edição n.º 02 – Editora Campus, Rio de Janeiro, Elsevier, 2003.

% VARIAN. Hal R, Microeconomia: Princípios Básicos. Edição n.º 06 – Editora Campus, São Paulo-SP, 2002.


% ----------------------------------------------------------
% Introdução
% ----------------------------------------------------------
\chapter{Introdução} 
Nas últimas décadas houve uma crescente demanda na área de processamento de dados, o que exigiu o desenvolvimento de máquinas
mais poderosas. Porém, apesar da grande capacidade existente nos novos computadores quando comparados aos primeiros a serem produzidos e até mesmo a computadores de uma década atrás principalmente ao se analisar capacidade de processamento e memória, eles ainda não são capazes de atender essas demandas de processamentos sozinhos. Por esse motivo foram desenvolvidas várias técnicas de concorrência o que contribuiu para a criação dos Sistemas Distribuídos.

%Um Sistema Distribuído é um sistema onde os componentes de \textit{hardware} ou \textit{software} se localizam em regiões físicas diferentes mas estão interligados em rede de forma que consigam se comunicar e coordenar suas ações entre sí.

Esse trabalho tem por objetivo praticar programação paralela usando o \textit{middleware} JavaRMI e realizar a análise de desempenho em um cluster de computadores. Ele consistirá na implementação de uma arquitetura mestre/escravo para realizar um ataque de dicionário em uma mensagem criptografada.


\chapter{Implementação} 
A implementação do trabalho foi feita utilizando a IDE NetBeans 8.2. Foram criados vários pacotes para facilitar a organização 
do código e das implementações de cada elemento. O trabalho é composto dos seguintes pacotes:

\begin{itemize}

	\item \textbf{br.inf.ufes.ppd:} Nesse pacote estão as interfaces padrões com os serviços oferecidos pelo escravo,
	cliente e mestre.
	
	\item \textbf{br.inf.ufes.ppd.application:} Nesse pacote estão as aplicações que oferecerão os serviços indicados nas  
	interfaces. Estão nesse pacote as aplicações de Cliente, Escravo, Mestre e a aplicação que realiza o servço de forma
	centralizada.

	\item \textbf{br.inf.ufes.ppd.implementation:} Nesse pacote se encontram as implementações dos serviços que são 
	fornecidos pelas aplicações do pacote anterior.
	
	\item \textbf{br.inf.ufes.ppd.tester:} No pacote \textit{tester} estão as aplicações criadas para obtenção automatizada
	das 	métricas de desempenho que foram utilizadas para a análise efetuada ao final do trabalho.
	
	\item \textbf{br.inf.ufes.ppd.utils:} Por fim, nesse pacote estão as funcionalidades de criptografia e descriptografia e 
	a de geração de dados em ".csv" para geração dos gráficos.
		

\end{itemize}

Para o correto funcionamento das aplicações desenvolvidas no trabalho sempre é necessário adicionar a seguinte diretiva ao comando de inicialização de cada elemento.

\begin{center}
	\begin{lstlisting}
		-Djava.rmi.server.hostname=(IP DA MAQUINA HOST)\end{lstlisting}
\end{center}

Essa diretiva é necessária para que o Java RMI possa adicionar a referência correta ao exportar um objeto remoto. 

Outro comando que deve ser executado antes da inicialização de qualque elemento é o \textit{rmiregistry}. Ele deve ser
executado dentro da pasta raiz onde se encontram as classes, no caso do NetBeans essa pasta é "build/classes/".
Nessa pasta se encontra a pasta raiz do pacote do trabalho, nesse caso a pasta "br".

\section{Estrutura de Dados}

Como o cliente é bem simples ele não possui uma estrutura muito complexa, já que basicamente ele só localiza um mestre para solicitar o serviço de ataque enviando logo em seguida o arquivo criptografado e o trecho conhecido ficando bloqueado enquanto não recebe uma resposta do mestre. Portanto, sua estrutura não será discutida em detalhes.

O escravo também não possuí muitas complexidades com relação a estrutura de dados, mas no seu serviço já existe um 
diferencial que é o uso de \textit{threads}. Basicamente, ao receber uma solicitação de subataque ele cria uma \textit{thread} para executar o subataque de forma que se outro ataque for solicitado ele não fica bloqueado atendendo ao primeiro que chegou.
A única informação que o escravo possui é seu ID de identificação que é único, e uma lista de chaves candidatas que é percorrida durante o ataque a fim de verificar se algumas delas produzem um arquivo que possua o trecho conhecido.
Outra \textit{thread} é usada no serviço de \textit{rebind} do escravo ao mestre.

A maior complexidade do trabalho foi na implementação do mestre. Como ele é o responsável pelo controle dos ataques, 
da distribuição dos subataques, por responder o cliente, entre outras tarefas, torna-se necessário o uso de uma estrutura 
de dados mais complexa para a realização de todas essas tarefas.

Sendo assim, para a implementação do mestre foram criadas as seguintes estruturas de dados:

\begin{itemize}

	\item \textbf{SubAttackControl:} Essa estrutura armazena todas as informações de controle de um subataque: uma variável 
	que diz se o ataque já terminou, o índice final do subataque e o último índice já verificado pelo escravo, através do
	\textit{checkpoint}.
	
	\item \textbf{AttackControl:} Essa estrutura é responsável por possuir as informações de um ataque solicitado por um
	cliente. Ele possui: a informação do tempo em que o ataque começou, possui uma variável que diz se o ataque está 
	terminado, uma referência para a mensagem criptografada e para o trecho conhecido daquele ataque e por fim uma lista 
	de subataques, que fazem parte daquele ataque.
	
	\item \textbf{SlaveControl:} Essa estrutura armazena as informações de cada escravo: nome, tempo do último checkpoint
	recebido e a lista de subataques que aquele escravo foi encarregado de executar.
	
\end{itemize}





\begin{itemize}


	\item \textbf{HashMap de UUID em Escravos:}
	
	\item \textbf{HashMap de AtaqueID em Lista de Guess:}
	
	\item \textbf{HashMap de SubataqueID em AtaqueID:}
	
	\item \textbf{HashMap de AtaqueID em AttackControl:}
	
\end{itemize}




\section{Decisões de Projeto}
Sub ataques

vetor separado de guess

vetor separado de ataques

o q fazer se nao tiver escravos

usar copia de slaves para iniciar ataque

a parada do checkpoint e do time


\section{Cliente}

O programa cliente recebe como argumentos: o nome do arquivo criptografado, o trecho conhecido do texto original e opcionalmente um terceiro parâmetro que indica o tamanho do vetor de \textit{bytes} aleatório que será gerado em caso do arquivo criptografado não existir.

Assim, o cliente é responsável por localizar o mestre utilizando o \textit{Registry} e solicitar o serviço de ataque através do método \textit{attack}. Ao solicitar o serviço o cliente passa o arquivo criptografado e o trecho conhecido. Caso o nome do arquivo fornecido como argumento para o programa cliente seja inválido o programa cliente é responsável por gerar um vetor aleatório de bytes, cujo tamanho será igual ao 3º parâmetro fornecido ou, caso esse não exista, um tamanho aleatório na faixa de 1Kb a 100Kb.

O comando utilizado para iniciar o cliente é:

\begin{lstlisting}
java br.inf.ufes.ppd.application.Client <Arquivo> <Trecho> [<Tam. vetor bytes>]\end{lstlisting}


\section{Escravo}
O escravo recebe como parâmetros: o caminho para o arquivo de dicionário, o nome do escravo que será criado e endereço do \textit{registry}. Caso alguns desses parâmetros não sejam fornecidos ele pega seus valores padrões existentes em um arquivo de configurações.

Inicialmente, o escravo deve procurar no \textit{registry} uma referência para o mestre, e assim que ele conseguir ele solicita ao mestre para ser adicionado através do serviço de \textit{addSlave} fornecido pelo mestre. Após estar registrado
no mestre o escravo fica aguardando alguma solicitação de subataque do mestre. 

O comando para se iniciar um escravo é:

\begin{lstlisting}
java br.inf.ufes.ppd.application.SlaveServer <dicionario> <nome> <registry>\end{lstlisting}



\section{Mestre}

O mestre recebe como parâmetro o endereço do \textit{registry}, caso ele não seja fornecido é utilizado o endereço padrão
que existe no arquivo de configurações.

O mestre é a aplicação que fornece o serviço de \textit{attack} para um cliente. Quando esse serviço é solicitado
ele delega subataques para os escravos que estão registrados nele. Assim, quando todos os escravos terminam seu trabalho o 
mestre envia a resposta para o cliente com os possíveis resultados encontrados durante os subataques.

O comando para se iniciar um escravo é:

\begin{lstlisting}
java br.inf.ufes.ppd.application.MasterServer <registry>\end{lstlisting}



\section{Problemas Durante a Implementação}
Durante o desenvolvimento do trabalho alguns problemas surgiram fazendo com que fosse necessário realizar algumas mudanças
na estrutura de dados e na lógica de programação. A maioria deles ocorreu durante o desenvolvimento do mestre, visto que
é o componente mais complexo do trabalho já que é responsável por gerenciar todos os escravos e ataques que ocorrem, além de 
atender as requisições que chegam dos cliente. 


\subsection{Cliente e Escravo}
O cliente foi o primeiro a ser implementado. Devido a sua simplicidade não foram encontrados problemas durante sua implementação.

Já na implementação do escravo, que também não era muito complicada, um problema ocorreu, apesar de 
ter sido de fácil solução.

O problema que surgiu foi durante a implementação do método \textit{startSubAttack}. Nos primeiros testes notou-se que os escravos não realizavam outros sub-ataques em paralelo, assim, ao analisar o código percebeu-se que não estavam sendo criadas \textit{threads} no método para realizar o ataque, dessa forma o escravo ficava bloqueado durante a execução de um ataque não atendendo outras requisições que chegavam. Ao se criar \textit{threads} esse problema foi solucionado.

Outro problema que surgiu durante a implementação do escravo foi o modo de verificar se o trecho conhecido existia no arquivo
descriptografado, podendo assim ser um candidato a ser o arquivo original. Essa verificação estava sendo feita da seguinte
forma: o array de bytes descriptografado era transformado em uma \textit{String} e a partir dela era usado o método dessa classe chamado \textit{contains}, que verifica se um uma \textit{substring} existe na \textit{string}. Porém ao se converter esse \textit{array} de \textit{bytes} em uma \textit{string} o Java utiliza uma codificação padrão interna que pode não ser a
mesma do arquivo que foi criptografado fazendo com que várias ou nenhuma chave candidata possa ser encontrada.

\subsection{Mestre}
O grande problema durante a implementação do mestre foi relacionado a estabelecer uma boa estrutura de dados para que fosse
possível realizar o gerenciamento dos escravos e ataques. No início do desenvolvimento a estrutura utilizada não possuía a ideia de subataques, isso fez com que surgisse um problema na hora da divisão de serviço entre os escravos. 

Como cada ataque possuía um número único de identificação ao dividir o trabalho entre os escravos eles recebiam esse mesmo número, assim, quando um escravo mandava um checkpoint, duas coisas estavam ocorrendo: 

\begin{itemize}

	\item Quando um escravo que recebia a primeira parte do serviço enviava um checkpoint ele era armazenado na estrutura de controle do ataque, porém quando um escravo que recebeu um intervalo de índices maior ao enviar o primeiro checkpoint já eliminava a informação do primeiro escravo, pois o checkpoint mais atual era salvo no lugar do anterior na estrutura de controle do ataque. Assim não se tinha uma forma de verificar o status de ataque de cada escravo, pois o que recebia o maior índice "dominava" nos checkpoints.

	\item A verificação para determinar se um trabalho havia terminado era baseada no índice recebido, se ele fosse igual ao último significava que havia terminado. Assim, quando o escravo que recebeu o maior índice terminava seu trabalho o mestre 
verificava que o serviço tinha acabado e enviava a resposta ao cliente, porém os outros escravos que estavam trabalhando no
mesmo ataque podiam não ter terminado a sua parte. Com isso, o mestre enviava uma resposta de um trabalho incompleto para o cliente.

\end{itemize}

A solução para resolver esse problema foi a criação de subataques para um ataque. Dessa forma, um ataque solicitado pelo cliente é dividido em vários subataques e cada escravo fica responsável por um, assim, ao enviar um checkpoint estará salvando em seu próprio subataque o andamento dos índices, fazendo com que o primeiro problema não ocorresse mais. Outro ponto, é que verificar se um trabalho terminou basta apenas que o mestre cheque se todos os subataques referentes àquele ataque foram terminados, isso garante que o mestre só irá enviar a resposta ao cliente quando todos os escravos tiverem terminado sua tarefa.

Outro problema que surgiu durante o desenvolvimento do mestre foi com relação a concorrência no acesso de variáveis. Existiram
algumas partes da estrutura que não estavam tendo sua execução serializada com o \textit{synchronized} de Java, e ao se adicionar vários clientes e escravos ocorreram exceções relacionados a concorrência. Um caso onde esse problema ocorreu foi quando o mestre ficava checando se um trabalho havia terminado para enviar a resposta ao cliente. Para realizar essa verificação o mestre ficava checando a todo o momento uma variável booleana que representava o \textit{status} do trabalho, se fosse falsa o trabalho não tinha terminado, caso contrário já tinha terminado. O problema era que quando o trabalho terminava e essa váriavel de \textit{status} ia ser modificada o mestre concorrentemente checava ela fazendo com que ocorresse exceção de acesso concorrente. Esse problema foi corrigido utilizando a opção \textit{wait} e \textit{notify} de Java que faz com que a \textit{thread} que verifica o \textit{status} do trabalho ficasse dormindo e quando o trabalho estivesse terminado ele era notificado e enviava a resposta para o cliente.

\chapter{Interoperabilidade}
A interoperabilidade do trabalho foi testada com os seguintes grupos:

\begin{itemize}

	\item Grupo: David Morosini e Gustavo Monjardim.
	\item Grupo: André Barreto e Eric Santos.
	\item Grupo: Eduardo e Gustavo.
	\item Grupo: Eduardo e Thiago Borges.

\end{itemize}

\section{Problemas Encontrados}
Um problema que aconteceu durante um dos testes de interoperabilidade foi no uso de um roteador, quando todo mundo estava conectado à rede um dos grupos abria o mestre e os outros criavam escravos, porém percebeu-se que havia uma demora de cerca de 2 a 3 minutos para que o mestre recebesse a solicitação e registrasse o cliente. E mais, ao se criar um ataque havia uma demora muito grande para que os escravos recebessem a solicitação do mestre, isso fez com que fosse inviável o uso do roteador para a realização dos testes.

Um outro problema que ocorreu foi que um grupo havia modificado a interface original adicionando um construtor com isso ocorria a exceção \textit{Unmarshalling Exception} que fazia com que não encontrasse as classes, pois em Java ao adicionar algum método a uma interface ele a considera uma nova versão diferente da original.

Problema dos indices

O restante dos problemas basicamente se relacionavam com o uso do registry

\chapter{Análise de Desempenho} 
Após a implementação do trabalho foram realizadas análises de desempenho com o objetivo de se conseguir observar se realmente
há uma melhoria ao se utilizar sistemas distribuídos. As análises estão nas seções seguintes.

\section{Máquinas e Equipamentos Utilizados}
Para a realização das análises foram utilizados dois tipos de computadores:

\begin{enumerate}

	\item Para a análise de desempenho de paralelismo em uma única máquina com vários escravos foi utilizado um notebook com processador \textit{Intel Core i5-2410M CPU 2.30GHz x 4} e 6GB de memória RAM com o Sistema Operacional \textit{Linux Mint KDE 64-bit}.

	\item Já para a análise de desempenho distribuído foram utilizados os computadores do laboratório de graduação (LabGrad) que possuem as seguintes configurações: processador \textit{AMD Athlon(tm) Dual Core Processor 5000B} com 4GB de memória RAM e o Sistema Operacional utilizado foi o \textit{Ubuntu 16.04.4 LTS}.

\end{enumerate}

Inicialmente, a ideia era realizar todos os testes nas máquinas do LabGrad porém essa divisão dos testes foi realizada pois ambos os LabGrads estavam reservados para aulas, e quando não estavam os outros alunos da graduação estavam utilizando, fazendo com que a disponibilidade de tempo nos laboratórios fosse bem pequena. 

Outra questão é que não seria possível testar o sistema em várias máquinas em casa já que só tinha disponível um computador, sendo assim, optei por fazer a análise de paralelismo no computador pessoal e deixar a análise de desempenho distribuído para ser realizada no LabGrad.

\section{Desempenho de Paralelismo em uma Única Máquina}

%\begin{figure}[!htb]
%\centering
%\caption{Legenda}
%\includegraphics[scale=0.5]{figuras/Hardware/pt4800.png}
%\label{fig:pt4800}
%\end{figure}


\section{Desempenho Distribuído em Várias Máquinas}

% Conclusão
% ---
\chapter{Conclusão}

Ao final desse trabalho é possível notar que um sistema distribuído é uma ferramenta muito poderosa pois permite aproveitar recursos de diferentes equipamentos para executar uma tarefa em comum.

Com a distribuição do serviço entre as máquinas é possível obter uma melhora significativa no tempo de resposta speedup. 
Observa-se ainda que não é possível atingir o speed up ideal segundo aLei de Amihdal devido ao overhead existente na rede, processamento, etc.



Outro ponto que é importante e que foi possível notar durante a implementação desse trabalho é a complexidade de funcionamento que existe na máquina responsável por realizar todo o gerenciamento, nesse caso representada pelo mestre. Ele tem que lidar: com o controle de máquinas ativas, com a distribuição das tarefas, com o controle do status das tarefas, com o controle de concorrência, redistribuição de tarefas, entre outros. Além de todas essas tarefas ele ainda tem que cuidar do acesso
concorrente à variáveis de controle, que é um grande problema ao se trabalhar com muitas \textit{threads}, assim, percebe-se a necessidade que sempre vai existir para que algum pedaço do programa não seja paralelizável.


falar das analises



% ----------------------------------------------------------
% ELEMENTOS PÓS-TEXTUAIS
% ----------------------------------------------------------
\postextual


% ----------------------------------------------------------
% Referências bibliográficas
% ----------------------------------------------------------
\bibliography{referencias} %% REFERENCIA AO ARQUIVO .bib

\end{document}
\grid