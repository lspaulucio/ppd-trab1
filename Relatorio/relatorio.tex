\documentclass[
	% -- opções da classe memoir --
	12pt,				% tamanho da fonte
	% openright,			% capítulos começam em pág ímpar (insere página vazia caso preciso)
    oneside,			% para impressão somente frente. Oposto a twoside (frente e verso)
	a4paper,			% tamanho do papel. 
	% -- opções da classe abntex2 --
	%chapter=TITLE,		% títulos de capítulos convertidos em letras maiúsculas
	%section=TITLE,		% títulos de seções convertidos em letras maiúsculas
	%subsection=TITLE,	% títulos de subseções convertidos em letras maiúsculas
	%subsubsection=TITLE,% títulos de subsubseções convertidos em letras maiúsculas
	% -- opções do pacote babel --
	english,			% idioma adicional para hifenização
	%french,				% idioma adicional para hifenização
	%spanish,			% idioma adicional para hifenização
	brazil,				% o último idioma é o principal do documento
	]{abntex2}


% ---
% PACOTES
% ---

% ---
% Pacotes fundamentais 
% ---
\usepackage{cmap}				% Mapear caracteres especiais no PDF
\usepackage{lmodern}			% Usa a fonte Latin Modern
%\usepackage{helvet}			% Usa a fonte helvet(ARIAL)
%\usepackage{pslatex}			
\usepackage[T1]{fontenc}		% Selecao de codigos de fonte.
\usepackage[utf8]{inputenc}		% Codificacao do documento (conversão automática dos acentos)
\usepackage{indentfirst}		% Indenta o primeiro parágrafo de cada seção.
\usepackage{color}				% Controle das cores
\usepackage{graphicx}			% Inclusão de gráficos
\usepackage{enumerate}

% ---
% Pacotes adicionais, usados no anexo do modelo de folha de identificação
% ---
\usepackage{multicol}
\usepackage{multirow}
% ---
	
% ---
% Pacotes adicionais, usados apenas no âmbito do Modelo Canônico do abnteX2
% ---
\usepackage{lipsum}				% para geração de dummy text
% ---

% ---
% Pacotes de citações
% ---
\usepackage[brazilian,hyperpageref]{backref}	 % Paginas com as citações na bibl
\usepackage[alf]{abntex2cite}	% Citações padrão ABNT

% --- 
% CONFIGURAÇÕES DE PACOTES
% --- 

% ---
% Configurações do pacote backref
% Usado sem a opção hyperpageref de backref
\renewcommand{\backrefpagesname}{Citado na(s) página(s):~}
% Texto padrão antes do número das páginas
\renewcommand{\backref}{}
% Define os textos da citação
\renewcommand*{\backrefalt}[4]{
	\ifcase #1 %
		Nenhuma citação no texto.%
	\or
		Citado na página #2.%
	\else
		Citado #1 vezes nas páginas #2.%
	\fi}%
% ---

% ---
% Informações de dados para CAPA e FOLHA DE ROSTO
% ---
\titulo{Relatório do 1º Trabalho de Processamento Paralelo e Distribuído}
\autor{Leonardo Santos Paulucio}
\local{Vitória - ES}
\data{22 de Maio de 2018}
\instituicao{%
  Universidade Federal do Espírito Santo
  %\par
  %Setor Palotina
  \par
  Engenharia da Computação}
\tipotrabalho{Relatório técnico}
% O preambulo deve conter o tipo do trabalho, o objetivo, 
% o nome da instituição e a área de concentração 
\preambulo{Trabalho apresentado à disciplina de Processamento Paralelo e Distribuído do curso Engenharia da Computação da Universidade Federal do Espírito Santo como requisito parcial de avaliação.
\newline \newline \textbf{Professor:} João Paulo A. Almeida}

% ---

% ---
% Configurações de aparência do PDF final

% alterando o aspecto da cor azul
%\definecolor{blue}{RGB}{41,5,195}
\definecolor{blue}{RGB}{0,0,0}

% informações do PDF
\makeatletter
\hypersetup{
     	%pagebackref=true,
		pdftitle={\@title}, 
		pdfauthor={\@author},
    	pdfsubject={\imprimirpreambulo},
	    pdfcreator={LaTeX with abnTeX2},
		pdfkeywords={abnt}{latex}{abntex}{abntex2}{relatório técnico}, 
		colorlinks=true,       		% false: boxed links; true: colored links
    	linkcolor=blue,          	% color of internal links
    	citecolor=blue,        		% color of links to bibliography
    	filecolor=magenta,      		% color of file links
		urlcolor=blue,
		bookmarksdepth=4
}
\makeatother
% --- 

% --- 
% Espaçamentos entre linhas e parágrafos 
% --- 

% O tamanho do parágrafo é dado por:
\setlength{\parindent}{1.3cm}

% Controle do espaçamento entre um parágrafo e outro:
\setlength{\parskip}{0.2cm}  % tente também \onelineskip

% ---
% compila o indice
% ---
\makeindex
% ---

% ----
% Início do documento
% ----
\begin{document}

% Retira espaço extra obsoleto entre as frases.
\frenchspacing 

% ----------------------------------------------------------
% ELEMENTOS PRÉ-TEXTUAIS
% ----------------------------------------------------------
% \pretextual

% ---
% Capa
% ---
\imprimircapa
% ---

% ---
% Folha de rosto
% (o * indica que haverá a ficha bibliográfica)
% ---
\imprimirfolhaderosto*
% ---

% ---
% RESUMO
% ---

% resumo na língua vernácula (obrigatório)
% \begin{resumo} %% AQUI COMEÇA A PÁGINA DE RESUMO
% Costuma-se dizer que, numa economia capitalista, os problemas econômicos relativos à decisão sobre que tipos de produtos devem ser produzidos e a que preços serão vendidos esses produtos são resolvidos normalmente pelo livre jogo das forças de mercado – isto é, pelo livre funcionamento da oferta e da demanda. Nesta hipótese, as decisões e escolhas econômicas são individualizadas e feitas pelos consumidores – que são os demandantes dos bens e serviços – e pelos produtores – que são os ofertantes. Agindo de acordo com seus próprios interesses, os indivíduos, afetando e sendo afetados pelo sistema de preços, tomam as decisões que maximizarão a satisfação coletiva. 
% O objetivo é o de explicar de maneira simplificada como atua um sistema de preços e sua influência na alocação de recursos escassos.
% Ocorre, porém, que a determinação do preço e da quantidade produzida de um bem ou serviço depende essencialmente do número de agentes econômicos – demandantes e ofertantes – existentes nesse mercado. Por isso, é interessante caracterizar, antes, os diversos tipos de mercado existentes.
% O mercado, como você sabe, é o local onde se encontram os vendedores e compradores de determinados bens e serviços. Antigamente, a palavra mercado tinha uma conotação estritamente geográfica, mas isso já está deixando de ser assim. Hoje, com os avanços tecnológicos nas comunicações, as transações econômicas podem se realizar sem contato pessoal direto entre comprador e vendedor, tal como ocorre nas compras e vendas pela internet.


%  \vspace{\onelineskip}
    
%  \noindent
%  \textbf{Palavras-chaves}: latex. abntex. editoração de texto.
% \end{resumo} %AQUI TERMINA A PÁGINA DE RESUMO
% ---

% ---
% inserir lista de ilustrações
% ---

%\listoffigures* %% o * indica que não será incluso no sumário
%\cleardoublepage %% Pula página
% ---

% ---
% inserir lista de tabelas
% ---

%\listoftables*
%\cleardoublepage
% ---

% ---
% inserir lista de abreviaturas e siglas
% ---
%\begin{siglas}
%  \item[Fig.] Area of the $i^{th}$ component
%  \item[456] Isto é um número
%  \item[123] Isto é outro número
%  \item[lauro cesar] este é o meu nome
%\end{siglas}
% ---

% ---
% inserir lista de símbolos
% ---
%\begin{simbolos}
%  \item[$ \Gamma $] Letra grega Gama
%  \item[$ \Lambda $] Lambda
%  \item[$ \zeta $] Letra grega minúscula zeta
%  \item[$ \in $] Pertence
%\end{simbolos}
% ---

% ---
% inserir o sumario
% ---

\tableofcontents*

% ---

% ----------------------------------------------------------
% ELEMENTOS TEXTUAIS  (necessário para incluir número nas páginas)
% ----------------------------------------------------------
\textual


% MONTELLA. MAURA. Micro e Macroeconomia: Uma Abordagem Conceitual e Prática. Edição nº 01 - Editora Atlas, São Paulo – SP, 2009.

% MANKIW, N. Gregory. Princípios da Microeconomia. Tradução Edição n.º 03 Norte-Americana – Editora Pioneira Thomson Learning, 2005.

% BEGG. David K. H. Introdução à Microeconomia. Edição n.º 02 – Editora Campus, Rio de Janeiro, Elsevier, 2003.

% VARIAN. Hal R, Microeconomia: Princípios Básicos. Edição n.º 06 – Editora Campus, São Paulo-SP, 2002.


% ----------------------------------------------------------
% Introdução
% ----------------------------------------------------------
\chapter{Introdução} 
Nas últimas décadas houve uma crescente demanda na área de processamento de dados, o que exigiu o desenvolvimento de máquinas
mais poderosas. Porém, apesar da grande capacidade existentes nos novos computadores quando comparados aos primeiros a serem produzidos e até mesmo a computadores de uma década atrás principalmente ao se analisar capacidade de processamento e memória, eles ainda não são capazes de atender essas demandas de processamentos sozinhos. Por esse motivo foram desenvolvidas várias técnicas de concorrência o que contribuiu para a criação dos Sistemas Distribuídos.

%Um Sistema Distribuído é um sistema onde os componentes de \textit{hardware} ou \textit{software} se localizam em regiões físicas diferentes mas estão interligados em rede de forma que consigam se comunicar e coordenar suas ações entre sí.

Esse trabalho tem por objetivo praticar programação paralela usando o \textit{middleware} JavaRMI e realizar a análise de desempenho em um cluster de computadores. Ele consistirá na implementação de uma arquitetura mestre/escravo para realizar um ataque de dicionário em uma mensagem criptografada.


\chapter{Implementação} 

\section{Estrutura de Dados}


\section{Problemas Durante a Implementação}
Durante o desenvolvimento do trabalho alguns problemas surgiram fazendo com que fosse necessário realizar algumas mudanças
na estrutura de dados e na lógica de programação. 


\subsection{Cliente e Escravo}
O cliente foi o primeiro a ser implementado. Devido a sua simplicidade não foram encontrados problemas durante sua implementação.

%O programa cliente recebe como argumentos: o nome do arquivo criptografado, o trecho conhecido do texto original e opcionalmente um terceiro parâmetro que indica o tamanho do vetor de \textit{bytes} em caso de o arquivo criptografado não existir.
%
%Assim, o cliente é responsável por localizar o mestre utilizando o \textit{Registry} e solicitar o serviço de ataque através do método \textit{attack}. Ao solicitar o serviço o cliente passa o arquivo criptografado e o trecho conhecido. Caso o nome do arquivo fornecido como argumento para o programa cliente seja inválido o programa cliente é responsável por gerar um vetor aleatório de bytes, cujo tamanho será igual ao 3º parâmetro fornecido ou, caso esse não exista, um tamanho aleatório na faixa de 1Kb a 100Kb.

Já na implementação do escravo, que também não era muito complicada, um problema ocorreu, apesar de 
ter sido de fácil solução.

O problema que surgiu foi durante a implementação do método \textit{startSubAttack}. Nos primeiros testes notou-se que os escravos não realizavam outros sub-ataques em paralelo, assim, ao analisar o código percebeu-se que não estavam sendo criadas \textit{threads} no método para realizar o ataque, dessa forma o escravo ficava bloqueado durante a execução de um ataque não atendendo outras requisições que chegavam. Ao se criar \textit{threads} esse problema foi solucionado.

\subsection{Mestre}
Os principais problemas que surgiram ocorreram durante a implementação do mestre, que é o responsável por gerenciar
todos os escravos e todos os ataques que estão ocorrendo no momento.


\chapter{Interoperabilidade}
A interoperabilidade do trabalho foi testada com os seguintes grupos:
\begin{itemize}
\item Grupo do David e
\item Grupo do André e Eric.
\item Grupo do Eduardo e Gustavo.   VER SE TIRA DEPOIS
\end{itemize}

\section{Problemas Encontrados}
falar da WIFI




\chapter{Análise de Desempenho} 
\section{Máquinas e Equipamentos Utilizados}
Nesse trabalho foi utilizado um computador com processador \textit{Intel Core i5-2410M CPU 2.30GHz x 4} com 6GB de 
memória RAM e o Sistema Operacional utilizado foi o \textit{Linux Mint KDE 64-bit}.


Pegar informações dos PCs do LabGrad.



\section{Uma Única Máquina}

\subsection{Centralizado}

\subsection{Paralelo com Vários Escravos}

\section{Várias Máquinas}

% Conclusão
% ---
\chapter{Conclusão}

Ao final desse trabalho é possível notar que um sistema distribuído é uma ferramenta muito poderosa pois permite aproveitar recursos de diferentes equipamentos para executar uma tarefa em comum.

Com a distribuição do serviço entre as máquinas é possível obter uma melhora significativa no tempo de resposta speedup. 
Observa-se ainda que não é possível atingir o speed up ideal segundo aLei de Amihdal devido ao overhead existente na rede, processamento, etc.



Outro ponto que é importante e que foi possível notar durante a implementação desse trabalho é a complexidade de funcionamento que existe na máquina responsável por realizar todo o gerenciamento, nesse caso representada pelo mestre. Ele tem que lidar: com o controle de máquinas ativas, com a distribuição das tarefas, com o controle do status das tarefas, com o controle de concorrência, redistribuição de tarefas, entre outros. Além de todas essas tarefas ele ainda tem que cuidar do acesso
concorrente à variáveis de controle, que é um grande problema ao se trabalhar com muitas \textit{threads}, assim, percebe-se a necessidade que sempre vai existir para que algum pedaço do programa não seja paralelizável.


 

falar das analises



% ----------------------------------------------------------
% ELEMENTOS PÓS-TEXTUAIS
% ----------------------------------------------------------
\postextual


% ----------------------------------------------------------
% Referências bibliográficas
% ----------------------------------------------------------
\bibliography{referencias} %% REFERENCIA AO ARQUIVO .bib

\end{document}
\grid
